%%%%%%%%%%%%%%%%%%%%%%%%%%%%%%%%%%%%%%%%%
% Large Colored Title Article
% LaTeX Template
% Version 1.1 (25/11/12)
%
% This template has been downloaded from:
% http://www.LaTeXTemplates.com
%
% Original author:
% Frits Wenneker (http://www.howtotex.com)
%
% License:
% CC BY-NC-SA 3.0 (http://creativecommons.org/licenses/by-nc-sa/3.0/)
%
%%%%%%%%%%%%%%%%%%%%%%%%%%%%%%%%%%%%%%%%%

%----------------------------------------------------------------------------------------
%	PACKAGES AND OTHER DOCUMENT CONFIGURATIONS
%----------------------------------------------------------------------------------------

\documentclass[DIV=calc, paper=a4, fontsize=11pt, twocolumn]{scrartcl}	 % A4 paper and 11pt font size
\usepackage[utf8]{inputenc}
\usepackage{lipsum} % Used for inserting dummy 'Lorem ipsum' text into the template
\usepackage[english]{babel} % English language/hyphenation
\usepackage[protrusion=true,expansion=true]{microtype} % Better typography
\usepackage{amsmath,amsfonts,amsthm} % Math packages
\usepackage[svgnames]{xcolor} % Enabling colors by their 'svgnames'
\usepackage[hang, small,labelfont=bf,up,textfont=it,up]{caption} % Custom captions under/above floats in tables or figures
\usepackage{booktabs} % Horizontal rules in tables
\usepackage{fix-cm}	 % Custom font sizes - used for the initial letter in the document

\usepackage{sectsty} % Enables custom section titles
\allsectionsfont{\usefont{OT1}{phv}{b}{n}} % Change the font of all section commands

\usepackage{fancyhdr} % Needed to define custom headers/footers
\pagestyle{fancy} % Enables the custom headers/footers
\usepackage{lastpage} % Used to determine the number of pages in the document (for "Page X of Total")

% Headers - all currently empty
\lhead{}
\chead{}
\rhead{}

% Footers
\lfoot{}
\cfoot{}
\rfoot{\footnotesize Page \thepage\ of \pageref{LastPage}} % "Page 1 of 2"

\renewcommand{\headrulewidth}{0.0pt} % No header rule
\renewcommand{\footrulewidth}{0.4pt} % Thin footer rule

\usepackage{lettrine} % Package to accentuate the first letter of the text
\newcommand{\initial}[1]{ % Defines the command and style for the first letter
\lettrine[lines=3,lhang=0.3,nindent=0em]{
\color{DarkGoldenrod}
{\textsf{#1}}}{}}

%----------------------------------------------------------------------------------------
%	TITLE SECTION
%----------------------------------------------------------------------------------------

\usepackage{titling} % Allows custom title configuration

\newcommand{\HorRule}{\color{DarkGoldenrod} \rule{\linewidth}{1pt}} % Defines the gold horizontal rule around the title

\pretitle{\vspace{-30pt} \begin{flushleft} \HorRule \fontsize{50}{50} \usefont{OT1}{phv}{b}{n} \color{DarkRed} \selectfont} % Horizontal rule before the title

\title{Sborník textů za rok 2014} % Your article title

\posttitle{\par\end{flushleft}\vskip 0.5em} % Whitespace under the title


\preauthor{\begin{flushleft}\large \lineskip 0.5em \usefont{OT1}{phv}{b}{sl} \color{DarkRed}} % Author font configuration

\author{Redaktoři portálu } % Your name

\postauthor{\footnotesize \usefont{OT1}{phv}{m}{sl} \color{Black} % Configuration for the institution name
www.asperger.cz % Your institution

\par\end{flushleft}\HorRule} % Horizontal rule after the title

\date{} % Add a date here if you would like one to appear underneath the title block

%----------------------------------------------------------------------------------------

\begin{document}

\maketitle % Print the title

\thispagestyle{fancy} % Enabling the custom headers/footers for the first page 

%----------------------------------------------------------------------------------------
%	ABSTRACT
%----------------------------------------------------------------------------------------

% The first character should be within \initial{}
\initial{A}\textbf{ni né po dvou měsících klidně můžeme vydat první sborník...}

%----------------------------------------------------------------------------------------
%	ARTICLE CONTENTS
%----------------------------------------------------------------------------------------


\subsection*{Soužití s aspergrem}

Vztah jako každý jiný, nebo vztah něčím jiný, než ostatní?

Návod na šťastné soužití neexistuje, bylo sepsáno plno příruček, jak
si partnera či partnerku udržet, jak být spokojený, jak šťastný, …

Popsat běžný den, NT a AS by nešlo. Každý den je něčím jiný. Každý NT
je jiný, a taktéž nejsou AS ve všem shodní. Tudíž je velmi
pravděpodobné, že nelze zobecnit jakési nástrahy, které v tomto
případě na vás mohou čekat. Nelze pojmout jednoho AS jako vzorek
všech. Dvě duše, dva lidé a jejich tak trochu jiný život…

Žiji s AS partnerkou 7 let. Jsme menšina – dvě ženy – v menšině. Tedy
statisticky. Není snadné najít člověka, se kterým chceme strávit celý
život. Být na stejné vlně, nemít žádné větší výhrady, respektovat a
být respektována. Důvěřovat. Věřit. Snít. Spolehnout
se. Mluvit. Mlčet. Těšit se na sebe.  Ale to platí o vztazích obecně.

Někdy mám pocit, že po probuzení nasadím jakési pomyslné rukavičky,
nasadím a mluvím jak s dítětem. Péťa má masivní úzkosti. A trpělivost,
která je třeba k ujišťování a opakování potřebuje rukavičky. Nesmí být
ovšem moc viditelné. Nesmí být hrané, nesmí být nucené. Cokoliv nejde
od srdce nemá význam. Má porouchané emoce. Má jich tolik, že neví co s
nimi. Vše, co je iracionální bolí, přináší množství somatickým
problémů. A mě bolí se dívat na to, jak trpí. Čemu můžu pomoci, lze
vůbec nějak předejít zbytečné bolesti? Bůh ví.

Příklad: iracionální strach ze zubaře, účetního, nákupu jídla třeba
pizzy tam kde ještě nebyla, fronta na poště a tlačítkový systém,
úřednice tam či onde….

Ke každé situaci se valí množství otázek. Za otázkou jde odpověď, pak
podotázka, varianta, vše jak rozvětvený strom který nemá konce. Z
každé větve další, menší, větší těžší, jiná se láme,…

Někdy se snažíme cestu někam, projít předem slovy, soustředit se na
možné situace, opakuje co říct, někdy je to jak divadlo. Skutečná
trasa cesty, slova, gesta, nákres atd., to se odehrává v obýváku jako
příprava. Vrchol našeho divadelního přestavení bylo jednou to, jak
předat lahev jako pozornost člověku XY.  Uleví se asi o procento, nebo
dvě. Ale hurá, někdy o víc. Někdy je lepší nepřemýšlet a konat. Bolí
ji to míň. Jak kdy. Nejhorší jsou neplánované akce (tím myslím třeba
dojít na VZP vyplnit nějaký formulář) nebo naopak dlouhodobě
plánované. Nevím si rady. Děláme soupisku. Nahlas ji
procházíme. Soupiska má-li víc jak tři úkoly začne zmatek. V tom
zmatku pak není místo na to myslet třeba na to – piju, jím? Mám hlad,
mám žízeň? Má otázky v hlavě. Má jich moc. Nejde myslet na pití. Vedle
stolu s PC má připravené 4 sklenice s vodou. Na očích. Všude je x
zápisníků. Poznámky, úkoly co je třeba. Na co nezapomenout. Vždy na
něco zapomene. Děláme si z toho srandu. Při mytí nádobí zůstane jedna
špinavá poklice. Ostatní je dokonale lesklé čisté, na místě. Poklice
jí splynula s okolím. Zapomenuta. Přehlídnuta. Na balkoně zůstane
deka. Prádlo. Kniha. Protože měla myslet třeba na pití.  Nebo naopak
je pyšná, že vzala domu prádlo, že nezapomněla, když začal takový obr
déšť. Že se tam máčí kniha na balkoně, mně už to nevadí. Rukavičky mi
buď přirostly k prstům, nebo jsem pochopila, že nutit k některým
činnostem nemá smysl. Nemá důvod připomínat to či ono. Je lepší
pochválit. Je lepší se smát než brečet. Vím, že se snaží, bojuje.

Jsem člověk, když mi rupnou nervy, a po x té připomínám pití, jídlo,
puštěné topení, teplé boty… někdy řeknu slovo, které by NT bral jen
jako slovo. Pro Péťu je to hádka, tragédie, něco velkého. Její popis
situace zpětně je plný expresivních výrazů. Hroutí se z
toho. Rozdýchává. A to jsem nikdy nezvýšila hlas do
křiku. Nikdy. Učíme se, že někdy křik je potřeba, že někdy malá hádka
vyčistí vzduch. Je třeba pochopit, že důvodem není urazit, poranit,
ale zlobím se, protože chráním, aby to či ono nebolelo. Nechat puštěný
vařič také není důvod ke klidu. Dát rychlovarnou konvičku na rozpálený
vařič, když to už udělal p. Tkaloun ve Vratných lahvích už není ani k
smíchu. Konvičku mi dali rodiče, je v tom nostalgie, jako když se vám
rozbije milovaný hrnek. Není dobré zvyšovat hlas, není dobré lpět na
věcech. Není dobré vyčítat. Z jedné výčitky Péťa má v hlavě strom
výčitek, strom otázek, tisíc slz, tisíc důvodů proč se mít méně
ráda. A já nechci, aby se měla méně ráda. Přeji si, ať se naučí mít se
ráda. Přeji si zázrak. Možná že ano. Možná že ne.

Máme úmluvu, že když cokoliv sní třeba poslední čokoládu, nebo můj
oblíbený salám, že to nevadí. Je dobré to říct a pak dokoupit. Nemůžu
se zlobit kvůli maličkostem. Pro mě je to maličkost. Pro ni Mount
Everest starostí. Zklamala. Ošidila mě. Vzala mi poslední sousto. Budu
se zlobit. Budu strááaašně nepříjemná. Rozejdeme se. Je mistr světa ve
zveličování. Mistr světa katastrof.  Jsem kecálek. Je problém pro mě –
mlčeti zlato.  Vše je o domluvě. Je třeba na situace se
připravovat. Znova si je procházet, mluvit o nich. Při snídani si
opakujeme, co dnes ji čeká jiného než je běžný plán. Má přesné časy na
venčení, na práci, na oběd atp. Když jsem odjela na pár dní pryč,
přiznala se, jelikož mi neumí lhát, že moc nejedla. Na pití
zapomínala. Nejedla skoro vůbec. Zapomněla. Až v noci ji hlad dohnal
anebo něco, co je kalorické snědla ve chvatu u počítače. Vítaly jsme
se několik hodin. Několik dní si říkaly, jak jsme si chyběly. Patříme
k sobě.

Sedm let žiji s partnerkou, která má AS, potkala jsem plno lidí, každý
jsme jedinečný.

Za její jedinečnost děkuji.  Bytost s  nádhernou duši, za nadlidský
spíše dětský údiv nad světem. Diví se lstivosti, lhaní, manipulování,
nemá ráda skrblíky, necity. Nechápe, proč tolik lidí jsou plní
lhostejnosti, křiku, zlosti. Chtěla by, aby svět fungoval
hezky. Většina z nás by chtěla nevyrůst a nemít odpovědnost a být
prcek. Jí se to si povedlo, myslím, že nikdy nebude velká. Má dar
zůstat malou holkou ve svém světě fantazie. Na úkor bolestných
somatizací, úzkostí a strachů.

V našem případě. AS + NT = láska na celý život. Z hloubi duše si to
přeji.

Autor: Lenka (NT)

Poznámka:K článku máme svolení k publikaci. K připsání této poznámky
mě přiměl stejný článek (ale také se svolením) na jiném blogu. 
This entry was posted in Ze života on 10.12.2014 by admin. Edit

\subsection*{Den Autismu – hodily by se nějaké zvyky?}

{\it Vážení a milí}

dovolte mi, abych Vám představil můj návrh na náš největší svátek,
kterým bezpochyby je Den autismu. Nejdříve jsem si myslel, že se to
nikomu moc nebude líbit, ale včera jsem mluvil s uživatelkou jménem
Depresorka, a říkala, že by to nebylo špatné.

Jelikož autisté mají zvyklosti, kterým se obvykle rodina autistů
přizpůsobuje, tak by se na tento den mělo udělat oblíbené jídlo, které
má ten daný autista rád. Tím chci říci, že pokud je někdo vybíravý a
nemá rád například Štědrovečerní večeři, tak by si mohl dát to jídlo,
které mu nejvíce chutná a je jedno, jestli je teplé či studené. Musíme
se řídit tím, že Den autismu vyznačuje den tolerance, nenucení apod.

Pro ty, kteří rádi okusí Štědrovečerní večeři, jsem si vymyslel
nějakou obdobu, která z této večeře vychází. Podle mně by měla večeře
vycházet z té Štědrovečerní, ale (a už jsme zase u toho), záleží co
komu chutná. A protože my Češi máme ke Štědrovečerní večeři kapra s
bramborovým salátem, tak jsem vymyslel obdobu. Byl by to plněný řízek
nějakou uzeninou, nejlépe šunkou či slaninou, nebo také sýrem. Také
mne napadla nějaká marináda. A co se týče bramborového salátu, měli by
být v něm některé suroviny stejné, ale aby to bylo trochu odlišné od
většinové společnosti, tak bych vymyslel nějaké, které tam nejsou, a
prosím Vás pište návrhy, za každý návrh budu rád. Můžete mi také něco
říci, pokud bych na něco zapomněl.

Také dále navrhuji, že by každý autista měl dostat na tento den nějaký
dárek. Také nejlépe ten, co má nejradši.

Dále bychom (jak už to tak děláme) mohli nosit v tento den modré
tričko či modré oblečení a vypustili bychom balónky. K tomu vypuštění
balónků jsem navrhl, že bychom si při tom mohli říci “Autisté
sobě”. Ale samozřejmě ne nahlas.

Také navrhuji uctění památky slavných autistů.

Pokud se tato moje myšlenky pošle aspoň trochu na veřejnost, tak budu
rád, ale pokud tato má myšlenka ovlivní autisty v ostatních státech
světa, tak přicházím i s myšlenkou, že by každé tradiční jídlo na Den
Autismu se jedlo podle regionů a vycházelo se z jídel, která jsou pro
ten daný region typická. Zvyky by mohli být stejné.

Děkuji.
This entry was posted in Úvahy aspíků on 8.12.2014 by
kipoisimotorku. Edit

\subsection*{Jak vnímám diskuze s textem: “Nic nedokážu, jsem úplnej blbec…”}

Hodněkrát se přidávám do diskuzí, kde někdo tvrdí takové věci, že je
blbý, že nic nedokáže. Tohle je snížené sebevědomí. Na tyto věty
odpovídám většinou jednou jedinou odpovědí. Že nikdo není blbý, že
všichni (minimálně aspíci :-)) dokážou alespoň jednu věc absolutně
výborně a nepřekonatelně.

Já umím pracovat s počítači, v tom mě hned tak někdo z NT
nepřekoná. Někdo umí vnímat věci v přírodě a ještě jiný umí pracovat v
účetnictví. Každý z aspergerů má jasně určené povolání prakticky hned
při svém zrodu. Ve svých genech mají podle mého skromného názoru
určeno hned od počátku co je bude bavit, co budou dělat. Tím se budou
také rádi živit, takže aspergeři mají vlastně hned při narození
určeno, co budou jednou dělat. Tím se také lišíme od NT. TI se mnohdy
rozhodují mnohdy až na úřadu práce. My to většinou víme už na základní
škole. To je podle mně jedna z velkých odlišností. Jasná představa
toho, čím by jsme se chtěli živit.

A co lidi, kteří mají poruchu soustředěnosti? Ti mohou (a velice často
i jsou) být užiteční jako stroj na nápady. Nápady nerealizují. Jen
přemýšlí a nápady, které vzniknou dají dál. Někdo jiný je pak
zrealizuje. Tyhle nápady jsou většinou skvělý. Jsou tak jednoduché, že
to prostě člověka ani nenapadne. A v tom to je, v jednoduchosti je
síla.

Jak to vnímáte vy?
This entry was posted in AS pohled na svět on 5.12.2014 by Martin
Urbanec. Edit

\subsection*{Jak jsem se stal depresivním mučedníkem a proč mi to (zatím) vyhovuje}


Dnes napíši hodně zvláštní a osobní článek, třeba mě pak lidi pochopí,
proč jsem, jaký jsem, a že jiný asi nebudu. A proč dělám, co dělám a
proč to dělám rád.

Neměl jsem ideální dětství, neměl jsem totiž diagnózu a tak mě lidé
nechápali, ve škole jsem nezapadal, vždy byl kolektiv a já. Doma to
taky nefungovalo, mamka přehnaně starostlivá, takže i když bylo
dětství díky tomu trošku snazší, ve výsledku jsem málo samostatný a
dnes je pro mě všechno velká zátěž, i když to má na svědomí i
Aspergerův syndrom. Taťka, ten byl pro změnu někdy hodný jako mamka, i
když přísný, ale někdy mu doslova přeskočilo a byl jak smyslů
zbavený. Měl jsem jednoho kamaráda, ale i tomu občas hráblo a byl
nevyzpytatelný. Takže dětství na prd, vždy jsem byl jako mimozemšťan
na cizí planetě.

Nejhorší byly ty tátovy záchvaty agresivity. To pak brečela mamka,
brečel bratr a brečel jsem i já, hodně jsme trpěli. Jednoho dne mě
napadlo, že ona ta mamka asi nebrečí až tak kvůli sobě, ale kvůli nám,
že trpíme. A tak jsem se rozhodnul, že ten pláč v sobě potlačím, aby
vypadalo, že se netrápím, aby se ani ona za mě netrápila. Jenomže víte
kolik to dá energie, potlačit to v sobě? Kdo to nezažil, neumí si to
představit. Když člověk ten hněv, smutek, zášť a kdo ví, jaké ještě
pocity v sobě dusí, tak se v něm hromadí a jsou jako žíravina, zevnitř
ho sežerou.

A tak jsem si postupně zablokoval dost citů, utvořil jsem v sobě
jakousi bariéru, nepustil jsem city dovnitř, ani ven, stal jsem se
chladným. Přestal jsem cítit lásku, přestal jsem cítit potřebu plakat,
jenom ta zášť a hněv vůči světu, který mi ublížil, zůstala a hromadila
se dál. To všechno bylo ještě před pubertou, to se mi nevyvíjela ještě
sexualita a tak.

Když puberta začala, já začal mít deprese, všichni spolu blbli,
tvořili partie a já nic, furt kolektiv a já. Neměl jsem se za to rád,
a to se zkombinovalo s tou záští a hněvem, co ve mně byl a já si začal
ubližovat, nebudu psát jak, bylo to špatné období a nechci na něj
tolik vzpomínat. No a jelikož se v pubertě rozvijí sexualita, a já
nenáviděl lidi a sebe, nějak se to ve mně pokazilo ještě víc a já se
stal masochistou. Jo, vzrušovalo mě, když jsem si jen představil, že
mi ubližují. Pak mi ale došlo, že vždyť já za nic z toho nemůžu, to mi
udělal svět, jiní lidé, hněv se obrátil ven a já byl pro změnu
sadista, toužil jsem ubližovat jiným. Ale jelikož jsem se za to hnusil
sám sobě, protože mám pořád dobré srdíčko a morálku, tak zůstal i hněv
na sebe, takže jsem byl exemplární případ milovníka BDSM.

Jo, koukal jsem i takové porno, líbilo se mi to, a hnusil jsem se za
to sám sobě, stal jsem se psychickou troskou. Stal jsem se apatický
vůči všemu, nic nemělo smysl, každý den jsem myslel jen na zvrácené
myšlenky a sebevraždu, chtěl jsem to zarazit. Ale neudělal jsem to, a
ani neudělám, i když ty myšlenky v sobě mám, nechci ublížit těm pár
lidem, co mě mají rádi, a kdo ví, třeba se to ještě všechno změní. A
taky by mi to nedovolilo mé ego, nechci umřít jako srab.

Tak jsem hledal způsoby, jak to ze sebe dostat ven, jak se přestat
nenávidět. Přišel jsem na to, řekl jsem si, že zasvětím svůj život
pomáháním druhým, já už jsem stejně troska a nechci, aby nevinní lidé
zažili to co já, jedna troska stačí. A taky jsem si uvědomil, že když
budu činit dobro, tak to napraví moji karmu za to, co zlé jsem chtěl
dělat druhým a že se mi to ještě líbilo. A ono to lidičky funguje,
zjistil jsem, že pomáhání druhým a jejich vděk za tu pomoc mě naplňuje
pozitivní energií, a i když jsou někteří lidé svině, já se přestal
zlobit na lidstvo, sadismus zmizel.

A tak teď trávím hodně času na Facebooku a radím každému, komu
umím. Zvolil jsem si život mučedníka, kašlu na sebe, ostatní potřebují
pomoc. Chtěl jsem vyhledat odbornou pomoc pro lidi s Aspergerovým
syndromem, ale těch odborníků je tady málo a mají plno, a proč bych
zabíral těm odborníkům čas, když jsou malé dětí, co potřebují pomoc
víc, než já. Já ten papír vlastně ani nepotřebuji, já ho jen chci, a
co tam po mně. Já mám divnou sexualitu, kupu fetišů, plno zášti a
chladnokrevnosti, ale naučil jsem se s tím žit, i když mám občas z
toho deprese. Beru se, jaký jsem. Tak či tak zůstanu nejspíš sám, a
možná to tak má být, třeba budu aspoň někoho hrdina.

Jsou lidé, kteří se mi to snaží vymluvit, abych myslel i trochu na
sebe a ano, kdybych se začal mít rád, tak by nejspíš zmizel i ten
masochizmus, ale já si na to natolik zvyknul, že je mi to
jedno. Nemyslím si, že v životě něco dosáhnu tak či onak, ani nevím,
co vlastně v životě chci. Takže prozatím na sebe kašlu, dokud si to v
hlavě neurovnám. Ale musím uznat, že slova chvály se dobře
poslouchají, avšak nestačí na to, abych se začal mít rád. Jsem prostě
na sebe tvrdý a lásku k sobě si musím sám před sebou zasloužit, jinak
bych si ji jen namlouval. Možná pak přijde změna, protože i když
neotravuji s prosbami o pomoc druhé, maximálně se občas vyzpovídám,
pracuji na sobě, učím se chápat sám sebe, tyto zkušenosti se samým
sebou pak používám na rady druhým, protože jsem zatím nenašel takový
oříšek, jakým jsem já sám…
This entry was posted in Úvahy aspíků, Ze života on 5.12.2014 by
Brankoslav. Edit
\subsection*{Klady a zápory mého modrého života – část 2. – ZŠ}
2 Comments

Mojí největší zálibou v předškolním věku bylo tzv. chození na
auta. Každý den, nehledě na počasí, jsem musel projít stejným směrem
všechna parkoviště na našem sídlišti. U každého auta jsem se musel
zastavit, abych nahlas přečetl SPZ a zjistil si, jaký typ poklic nebo
ráfků má dané auto, dokonce jsem si je nazýval nesmyslnými názvy.
Jednalo se o každodenní rituál, který trval něco přes hodinu a nesměl
se vynechat. Z dovolené i z výletů jsme se vraceli tak, abych se stihl
jít s jedním z rodičů podívat na auta. Takto to trvalo neskutečně
dlouhé 3 roky a děkuji rodičům za velkou trpělivost. Dalším rituálem
bylo stavění auta ze všech kostiček lega, co jsem měl.

Na jaře r. 2004 jsem absolvoval vyšetření u PhDr. Thorové, která mi
diagnostikovala AS. Téhož roku jsem nastoupil na ZŠ. Ve škole se mi
začalo líbit v okamžiku, kdy jsem zjistil, že tam nemusím po obědě
spát. V té době jsem miloval Písničky z pohádek a dětských filmů,
které p. uč. hrála často na kytaru, a díky tomu jsem se ve škole cítil
dobře. Měl jsem štěstí na paní asistentku, která mi rozuměla a plně
chápala mé problémy. Snažila se mně maximálně přiblížit, počítala se
mnou slovensky i maďarsky, učila se latinské názvy zvířat, hub
atd. Pamatuji si, že jsem si v 1. třídě se spolužáky rozuměl, dávali
jsme si dárečky, malovali obrázky a tak dále, prostě znáte to.

Problémy začaly ve 4. třídě, kdy jsem nemohl unést to, že některým
dětem začala puberta a p. uč. se na ně zlobila. Já měl rád jí i děti a
jejich spory mě těžce nervovaly. Nevěděl jsem na čí straně stát a
bohužel jsem si pak vybral svět dospělých, což, jak už dávno vím, jsem
udělal špatně a popudil jsem proti sobě hodně spolužáků :(. V
5. třídě, tedy na podzim 2008, se objevily případy žloutenky a já měl
takový strach, že mi to přerostlo v panickou hrůzu z nemocí, které by
mi ani nehrozily. První stupeň mi skončil tak, že už ty vztahy nebyly
takové jako dřív, ale později jsem si uvědomil, že jsem si za to mohl
částečně sám. Dlouhou dobu jsem si to vyčítal, ale až teď, tedy po
skončení ZŠ, jsem se potkal s pár lidmi z prvního stupně a zjistil
jsem, že někteří z těchto lidí mají postoj (sice s menšími změnami)
stejný. Projevuje se to např. tím, že se dokáží se mnou normálně
bavit, což hodně lidí z druhého stupně nedokáže. Sice ne všichni lidi
z prvního stupně (zvláště ten spolužák, kterého jsem napomínal) se se
mnou nebaví jako s obyčejným člověkem, ale je i taková část, která má
větší odstup ode mne. Holt co jsem si nadrobil, tak jsem si
sežral. Toť k prvnímu stupni.

No a na závěr článku je ještě jedno poučení, které zní: “Aby si člověk
mohl všecko uvědomit, musí padnout na dno.” A skutečně se tak
stalo. To dno se jmenovalo druhý stupeň.

 
\subsection*{Měsíční zpráva portálu asperger.cz}


Máme za sebou první z doufám mnoha měsíců (doufám, že i let) existence
portálu aspie.cz, což je čas na první měsíční zprávu.

Určitě první otázka by byla, jakou máme návštěvnost? Hned na ni
odpovím pomocí přehledu z našeho sledovacího nástroje, který si můžete
stáhnout (PDF, 159 kB).

Jaká je úspěšnost našeho portálu? Vzhledem k tomu, že existujeme
pouhopouhý jeden měsíc, tak bych řekl, že je nadpřirozeně
vysoká. Hodně nadpřirozeně. Popravdě jsem nečekal, že získávání autorů
a dalších přispěvovatelů půjde tak snadno. Náš rozpočet vypadá velice
dobře. Můžete si ho stáhnout (PDF, 73 kB). To všechno během jednoho
měsíce.

Myslím, že aspíci prostě chtějí mít něco takového. Proto všechno šlo
tak hladce. Když se o něco takového pokoušel někdo jiný před nějakou
dobou, prostě mu to nešlo. V té době asi aspíci mít něco takového
nechtěli. Nebo to je něčím jiným.

Zvláštní poděkování patří těmto lidem:

    Michalu Struhárovi, bez jeho ťuknutí by nápad a ani nic takového
    nevzniklo, jeho nápaditost je neobyčejně vysoká
    Branislavu Lackovi, za jeho vysoký příspěvek na fungování webu a
    podporu v FB skupinách zaměřených na Aspergerův syndrom
    Depresorce, za její negativní pohled na svět, stáváme se díky ní
    pestrým portálem

Poděkování patří také všem redaktorům, všem, co přispěli k fungování
webu i všem čtenářům.


%----------------------------------------------------------------------------------------
%	REFERENCE LIST
%----------------------------------------------------------------------------------------

%\begin{thebibliography}{99} % Bibliography - this is intentionally simple in this template

%\bibitem[Figueredo and Wolf, 2009]{Figueredo:2009dg}
%Figueredo, A.~J. and Wolf, P. S.~A. (2009).
%\newblock Assortative pairing and life history strategy - a cross-cultural
%  study.
%\newblock {\em Human Nature}, 20:317--330.
 
%\end{thebibliography}

%----------------------------------------------------------------------------------------

\end{document}
